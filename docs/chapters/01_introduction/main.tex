\chapter{Introduction}
\label{chap:introduction}
\setlength{\parskip}{1em}

We, as developers with focus to AI, wanted to research on improving efficiency in manufacturing facilities. Cutting unplanned stoppages in factories is essential for peak productivity, so we wanted to build a machine learning model to fulfill this goal. Recent breakthroughs in machine learning—especially in time-series analysis and anomaly detection—mean it’s now realistic to spot the earliest signs of wear or failure in pumps, motors and other equipment. These methods can run on data you already collect (temperatures, pressure, electrical loads) and flag trouble well before a breakdown halts production.
Today’s market offers plenty of “smart” maintenance platforms, but most are closed-source, costly to customize and hard to audit. That makes it understanding errors and warnings difficult for engineers, or to adapt models to the quirks of their own lines. Relying on a black-box system can leave teams guessing whether alerts are real, or just another false alarm.
Our goal is to change that by developing transparent, open-source predictive models tailored to industrial settings. We’ll start by gathering sample datasets from real machines, then compare several machine-learning approaches—like long short-term memory, convolutional neural network and recently popularized transformer models. Along the way, we’ll document every step, from data cleaning to model validation, so other engineers can follow, tweak and trust our work.
By sharing our code, models and findings, we hope to give manufacturers a reliable alternative: a toolbox they can inspect, adapt and extend without vendor lock-in. In the end, we want to turn every factory’s own data into an early warning system that’s as open and flexible as the shop floor itself.