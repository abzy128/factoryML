\chapter{Literature Review}
\label{chap:literature}
\setlength{\parskip}{1em}

The integration of advanced technologies in manufacturing, commonly referred to as Industry 4.0, is transforming industrial processes. This study explores the development of models for equipment sensors in industrial facilities using machine learning techniques, based on the available literature.
The discussion will focus on key topics such as Industry 4.0, smart manufacturing, equipment sensor technology, machine learning applications, time series data processing methods, predictive maintenance techniques, and ethical considerations for the deployment of industrial artificial intelligence.

\section*{1. Industry 4.0 and Smart Manufacturing}

Industry 4.0 represents a paradigm shift in manufacturing, characterized by the integration of cyber-physical systems, the Internet of Things (IoT), and advanced data analytics \cite{zhang-2021}. Zhang et al. \cite{zhang-2021} provide a comprehensive review of Industry 4.0 and its implementation, highlighting its potential to enhance efficiency and productivity in manufacturing.  The concept of smart manufacturing, a key component of Industry 4.0, leverages interconnected systems and data-driven decision-making to optimize production processes. Liu et al. \cite{liu-2023} delve into the IoT ecosystem for smart predictive maintenance (IoT-SPM) in manufacturing, emphasizing the multiview requirements and data quality crucial for effective implementation. Their evaluative study underscores the importance of a robust IoT infrastructure for realizing predictive maintenance capabilities.

Digital twins, virtual representations of physical assets, are also integral to smart manufacturing. Lattanzi et al. \cite{lattanzi-2021} review the concepts of digital twins in the context of smart manufacturing, exploring their practical industrial implementation. Digital twins facilitate real-time monitoring and simulation, enabling proactive maintenance and process optimization. Furthermore, the principles of Industry 4.0 extend to sustainability in manufacturing. Awasthi et al. \cite{awasthi-2021} discuss sustainable and smart metal forming manufacturing processes, indicating the broader impact of these technological advancements on environmental and economic aspects of production.

\section*{2. Equipment Sensors in Manufacturing Facilities}

Equipment sensors are fundamental to acquiring real-time data in manufacturing environments, enabling monitoring, control, and optimization of industrial processes. Jiang et al. \cite{jiang-2020} present a review on soft sensors, which are inferential sensors that utilize readily available process measurements to estimate difficult-to-measure variables. These sensors are crucial for enhancing process visibility and control.  For effective condition monitoring, robust data acquisition systems are essential. Toscani et al. \cite{toscani-2023} introduce a novel scalable digital data acquisition system designed for industrial condition monitoring, highlighting its potential for real-time data collection and analysis.

The data collected from equipment sensors, often in the form of multivariate time-series data, plays a critical role in detecting anomalies and predicting equipment failures. Nizam et al. \cite{nizam-2022} propose a real-time deep anomaly detection framework specifically for multivariate time-series data in industrial IoT settings. Their work demonstrates the application of deep learning for timely anomaly detection, which is crucial for preventing downtime and ensuring operational continuity. Pech et al. \cite{pech-2021} further emphasize the role of predictive maintenance and intelligent sensors in the smart factory, providing a review of how these technologies converge to create more efficient and resilient manufacturing systems.

\section*{3. Machine Learning in Industrial Applications}

Machine learning (ML) is at the core of analyzing sensor data and developing predictive models for industrial applications. Amer et al. \cite{amer-2023} discuss the application of machine learning methods for predictive maintenance, showcasing how ML algorithms can be trained to predict equipment failures based on sensor data.  Anomaly detection, a key application of ML in manufacturing, is further explored by Liu et al. \cite{liu-2021}. They propose an anomaly detection method on attributed networks using contrastive self-supervised learning, which can be adapted for identifying unusual patterns in sensor networks.

In the context of industrial soft sensors, Ou et al. \cite{ou-2022} introduce quality-driven regularization for deep learning networks. Their work focuses on enhancing the reliability and accuracy of soft sensors through advanced deep learning techniques.  Addressing the challenge of limited data in industrial settings, Zhou et al. \cite{zhou-2022} present a time series prediction method based on transfer learning. This approach is particularly relevant in manufacturing environments where historical failure data might be scarce, enabling more effective predictive modeling even with limited datasets.

\section*{4. Time Series Data Processing for Equipment Monitoring}

The data generated by equipment sensors is typically time-series data, requiring specialized processing techniques for effective analysis and prediction. Islam et al. \cite{islam-2024} introduce a novel probabilistic feature engineering approach, RKnD, for understanding time-series data, although their specific application is in driver behavior understanding, the principles of feature engineering are transferable to manufacturing sensor data.  Makridakis et al. \cite{makridakis-2022} provide a comprehensive comparison of statistical, machine learning, and deep learning forecasting methods for time series data. Their review offers insights into the strengths and weaknesses of different methods, guiding the selection of appropriate techniques for equipment monitoring.

Preprocessing sensor data to remove noise and enhance signal quality is crucial for accurate analysis. Alami and Belmajdoub \cite{alami-2024} discuss noise reduction techniques in sensor data management, although focused on ADAS sensors, the comparative analysis of methods is relevant for industrial sensor data as well.  Furthermore, understanding the context of the manufacturing process is important. Taskinen and Lindberg \cite{taskinen-2024} highlight the challenges facing non-ferrous metal production, providing a domain-specific perspective that can inform the development of sensor-based monitoring systems in metal smelting factories.

\section*{5. Predictive Maintenance in Metal Smelting Factories}

Predictive maintenance is a critical application of sensor-based monitoring and machine learning in industries like metal smelting. Olesen and Shaker \cite{olesen-2020} present a state-of-the-art review of predictive maintenance for pump systems and thermal power plants, outlining trends and challenges that are also pertinent to metal smelting factories which often involve similar equipment. Leukel et al. \cite{leukel-2021} systematically review the adoption of machine learning technology for failure prediction in industrial maintenance. Their findings are valuable for understanding the practical implementation and benefits of ML-based predictive maintenance strategies.

The economic evaluation of implementing artificial intelligence in manufacturing, including predictive maintenance, is also a key consideration. Chen et al. \cite{chen-2021} discuss the economic evaluation of energy efficiency and renewable energy technologies using artificial intelligence, providing a framework for assessing the financial viability of AI-driven solutions in industrial settings.

\section*{6. Challenges and Ethical Considerations in Industrial AI Deployment}

Deploying AI and machine learning models in industrial environments is not without challenges and ethical considerations. Khowaja et al. \cite{khowaja-2022} propose a two-tier framework for data and model security in industrial private AI, addressing the critical aspect of data privacy and security in interconnected manufacturing systems. Paleyes et al. \cite{paleyes-2022} provide a survey of case studies highlighting the challenges in deploying machine learning in real-world applications, emphasizing the practical hurdles that need to be overcome. Landers and Behrend \cite{landers-2022} discuss the ethical dimension, specifically focusing on auditing AI auditors and evaluating fairness and bias in high-stakes AI predictive models, raising important questions about the responsible and ethical deployment of AI in manufacturing.